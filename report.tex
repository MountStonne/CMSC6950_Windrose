
% This is the main file to setup the document.
% Document organization and appearance settings are all done here
% Each chapter is a separate tex file, all linked together here


% Preamble (document settings) -----------------------------------------------------------
% Document type and font --
\documentclass[10pt]{report}
\usepackage[utf8]{inputenc} %utf-8 encoding for ASCII symbols

% insert packages here --
\usepackage{graphicx}       %for handling images

\usepackage{amsmath}        %for math symbols

\usepackage{breakcites}     %to avoid citations extending into the margin

\usepackage[margin=1in]{geometry}   %to reduce margins to 1 inch, default margins wasted a lot of space

\usepackage{sidecap}        %to enable side captions on figures

\usepackage{setspace}       %to enable doublespacing   

\usepackage[american]{babel}
\usepackage{csquotes}
\usepackage[style=apa]{biblatex}
\DeclareLanguageMapping{american}{american-apa}
%\usepackage[backend=biber,style=apa,citestyle=apa]{biblatex}       %use the biblatex package

\addbibresource{mybibfile1.bib}   %path to the bib file


\usepackage{hyperref}       % to create a linked table of contents
\hypersetup{colorlinks,citecolor=black,filecolor=black,linkcolor=black,urlcolor=black}

% Set path to images
\graphicspath{ {images/} }  % Direct to the main image folder, always good to create sub-folders to organize images for individual chapters

\singlespacing  %making text double spaces

% End of preamble
%-------------------------------------------------------------------------------------------
\begin{document}

% Making title page
\begin{titlepage}
   \begin{center}
   \begin{doublespacing}

       \begin{figure}
       \centering
       \includegraphics[width=0.7\textwidth]{images/MUN_Logo_RGB.png}
       \end{figure}
       
       
       \vspace*{5mm}
       {\large\textbf{CMSC6950 COURSE PROJECT}}

       %\textbf{DOCTOR OF PHILOSOPHY}
       \vspace{30mm}
       
       {\Large\textbf{WINDROSE APPLICATION}}

    
            
       \vspace{30mm}

       {\Large\textbf{SHUO WANG}\\
       %\textbf{}
       }

       \vfill
       {\large \textbf{Department of Computer Science}\\
       \textbf{August, 2020}}
       
    \end{doublespacing}

   \end{center}
\end{titlepage}


% Starting frontmatter:
% Abstract goes here
\doublespacing
 \thispagestyle{plain}
\begin{center}
    \Large
    \textbf{Windrose Application}
        
    \vspace{0.4cm}
    \textbf{Shuo Wang}
    
    \vspace{0.4cm}
    \large{Supervisor: Prof James Munroe}
       
    \vspace{0.9cm}
    \textbf{Abstract}
\end{center}

Wind rose is a graphic tool to describe the wind speed and direction distribution at a particular location. In this project, wind data of St. John's processed by Numpy and Pandas, wind rose was plotted by Matplotlib. After five different plots was analysed, the result indicated that most wind in St. John's is from northwest and southeast, and the speed is range from 0.05 m/s to 19.62 m/s, which will be useful for environmental research and infrastructure construction.

\pagenumbering{roman}   % Roman page numbering to start from abstract onwards

\singlespacing          % keep pre-content single spaced
\listoffigures          % generate list of figures
% \listoftables           % generate list of tables

% End of frontmatter

% Insert table of contents
 \tableofcontents

% Main matter starts here --
% Inserting individual chapters. Mention chapter titles here and simple link the chapter's tex file

 \chapter{Introduction}     % Mention chapter title here
  \pagenumbering{arabic}    % We want Arabic numerals for main matter page numbering
 
 \section{Wind rose}

A wind rose is a graphic tool used by meteorologists to give a succinct view of how wind speed and direction are typically distributed at a particular location. Historically, wind roses were predecessors of the compass rose, as there was no differentiation between a cardinal direction and the wind which blew from such a direction(\cite{wikipedia}). \par Using a polar coordinate system of gridding, the frequency of winds over a time period is plotted by wind direction, with color bands showing wind speed ranges. The direction of the longest spoke shows the wind direction with the greatest frequency as shown below. \par

\begin{figure}[h!]
    \centering
    \includegraphics[width=0.50\textwidth]{images/example.png}
    \caption{An example wind rose (\cite{Roubeyrie2018})}
    \label{fig: PaleBlueDot}
\end{figure}

% \textcite{parish2009propagation}

\section{Data source}
The wind data of St. John's was obtained directly from the SmartAtlantic Alliance which is an initiative of the Fisheries and Marine Institute of Memorial University of Newfoundland's Centre for Applied Ocean Technology (CTec) and the Centre for Ocean Ventures and Entrepreneurship (COVE) of Halifax.\par
The URL of CSV data file was generated from their website as the figure shown below. \par

\begin{figure}[h!]
    \centering
    \includegraphics[width=0.50\textwidth]{images/data_source.png}
    \caption{URL generation of CSV data file}
    \label{fig: PaleBlueDot}    
\end{figure}

In this project, the information of timestamp, wind speed and wind direction in the last 365 days (from 1 August 2019 to 31 July 2020) was collected and used to plot wind rose.\par

\section{Analysis workflow}
After collecting the raw wind data, it was processed by the methods from python package Numpy and Pandas. The details of what processes have been done will be discussed in the next chapter.\par
The processed wind data was used to implement the visualization and five different graphs were drew with python package Matplotlib tools.\par
As plotting the graphs, the wind distribution of St. John's reflected from the images was also discussed.


\chapter{Data management and analysis}
In this chapter the raw wind data was downloaded, processed and regenerated for plotting.

\section{Data process by Pandas}

\subsection{Read raw data}
The raw data imported to a dataframe by \texttt{pandas.read\_csv} from the URL generated before has a size of 44338 rows $\times$ 3 columns. The first several lines were shown in figure below.

\begin{figure}[h!]
    \centering
    \includegraphics[width=0.50\textwidth]{images/raw_data.png}
    \caption{The first several lines of raw wind data}
    \label{fig: PaleBlueDot}    
\end{figure}

\subsection{Adjust data format}
The right two columns were renamed to "speed" and "direction" by \texttt{pandas.DataFrame.rename} and the left one column was set to the index of the dataframe by \texttt{pandas.DataFrame.set\_index}. Then the send row which is the unit was removed by \texttt{pandas.DataFrame.drop}. The dataframe type was converted to float64 by \texttt{pandas.DataFrame.astype} for the following calculations.\par
After the adjustment, the dataframe was more clear as shown in figure below.\par
\begin{figure}[h!]
    \centering
    \includegraphics[width=0.50\textwidth]{images/processed_dataframe.png}
    \caption{Dataframe after adjustment}
    \label{fig: PaleBlueDot}    
\end{figure}
\subsection{Generate speed data on x and y axis respectively}
The wind speed on x and y axis was calculated by the following functions:\par
\[Speed_x = speed \times \sin(direction \times \pi / 180)\]
\[Speed_y = speed \times \cos(direction \times \pi / 180)\]
After the generation, there were two more columns in the dataframe as shown in the figure 2.3 below.\par
\begin{figure}[h!]
    \centering
    \includegraphics[width=0.50\textwidth]{images/processed_dataframe_xy.png}
    \caption{Dataframe with speed on x and y axis respectively}
    \label{fig: PaleBlueDot}    
\end{figure}


\section{Remove none and zero values}
A python function called \texttt{clean\_df} was generated to remove the none and zero values. After that, a new CSV file was generated by \texttt{pandas.DataFrame.to\_csv} as shown below.
\begin{figure}[h!]
    \centering
    \includegraphics[width=0.50\textwidth]{images/processed_data.png}
    \caption{The first several lines of processed data}
    \label{fig: PaleBlueDot}    
\end{figure}

 \chapter{Data Visualization}
 In this chapter five plot was analysed in order to summary the wind condition of St. John's.

\section{Cartesian coordinate system plot}

\subsection{A basic scatter plot with transparency}
\begin{figure}[h!]
    \centering
    \includegraphics[width=0.50\textwidth]{images/figure1.png}
    \caption{Wind speed distribution of St. John's in the last one year}
    \label{fig: PaleBlueDot}    
\end{figure}
The figure above indicated that the wind speed on X axis ranged from about -15 m/s to 8 m/s, on Y axis ranged from about -17 m/s to 12 m/s in the last whole year.
\section{Wind rose}
 \subsection{Wind rose like a stacked histogram with normed results}
 \begin{figure}[h!]
    \centering
    \includegraphics[width=0.50\textwidth]{images/figure2.png}
    \caption{Wind rose like a stacked histogram with normed results}
    \label{fig: PaleBlueDot}    
\end{figure}
The wind rose below indicated that most wind directions are northwest and southeast and account for over 50 percent.

 \subsection{Another stacked histogram representation, not normed, with bins limits}
 \begin{figure}[h!]
    \centering
    \includegraphics[width=0.50\textwidth]{images/figure3.png}
    \caption{Another stacked histogram representation, not normed, with bins limits}
    \label{fig: PaleBlueDot}    
\end{figure}
The second figure below shows the exact values of wind speed instead of percentage.

\subsection{A wind rose in filled representation, with a controled colormap}
 \begin{figure}[h!]
    \centering
    \includegraphics[width=0.50\textwidth]{images/figure4.png}
    \caption{A wind rose in filled representation, with a controled colormap}
    \label{fig: PaleBlueDot}    
\end{figure}
Compared to the two figures above, this wind rose is more consecutive. Wind speed information can be obtain from any point in this figure.

\subsection{A wind rose in filled representation, with contours}
 \begin{figure}[h!]
    \centering
    \includegraphics[width=0.50\textwidth]{images/figure5.png}
    \caption{A wind rose in filled representation, with contours}
    \label{fig: PaleBlueDot}    
\end{figure}
After adding the contours, the wind rose was visualized better.
 
 \chapter{Conclusion}
 The graph results shows that more than 50 percent wind in St. John's is from northwest and southeast. The maximum wind is in the last year is 19.62 m/s, while the minimum wind is m/s. The wind direction and speed may change with season changed which will be explored in the future studies.

% Insert bibliography here
\printbibliography

\end{document}

% End of document
%-----------------------------------------------------------------------------------------------------